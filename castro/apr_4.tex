\documentclass[12pt,letterpaper]{hmcpset}
\usepackage[margin=1in]{geometry}
\usepackage{graphicx}
\usepackage{amsmath,amssymb}
\usepackage{enumerate}

% info for header block in upper right hand corner
\name{}
\class{Math 45 - Section ---}
\assignment{Homework 3}
\duedate{Monday, April 4, 2016}

\newcommand{\RR}{\mathbb{R}}

\begin{document}

\problemlist{C.\{1,2,3,4,5,6,7,8\}}

\begin{problem}[C1]
    Find the equilibrium points of the following
    differential equations and determine their 
    stability:
    \begin{enumerate}[(a)]
        \item $y' = y(3-y)-1$
        \item $y'=y^3(y^2-1)$
    \end{enumerate}
\end{problem}

\begin{solution}
    \vfill
\end{solution}
\clearpage

\begin{problem}[C2]
    Find the equilibrium points of the following
    differential equations and determine their 
    stability:
    \begin{enumerate}[(a)]
        \item $y' = y^2(y^2-1)$
        \item $y' = e^{-y}\sin(y)$
    \end{enumerate}
\end{problem}

\begin{solution}
    \vfill
\end{solution}
\clearpage

\begin{problem}[C3]
    For each of the following find an equation 
    $y' = f(y)$ with each of the following stated
    properties. If there no examples explain why not.
    (In all cases assume that f(y) is continuously
    differentiable.)
    \begin{enumerate}[(a)]
        \item Every real number is an equilibrium point.
        \item Every integer is an equilibrium point and
            there are no others.
        \item There are no equilibrium points.
        \item There are exactly 100 equilibrium points.
    \end{enumerate}
\end{problem}

\begin{solution}
    \vfill
\end{solution}
\clearpage

\begin{problem}[C4]
    Find the values of $p > 0$ such that the solution
    to $y'=-y^p$, $y(0)=1$ satisfies satisfies $y(t) =
    0$ for some $t > 0$. Use the existence and uniqueness
    theorem to prove that if $p\geq 1$ then $y(t) > 0$
    for all $t\in[0,+\infty)$.
\end{problem}

\begin{solution}
    \vfill
\end{solution}
\clearpage

\begin{problem}[C5]
    Suppose the natural growth of a population is $.02$
    and that there is no competition for resources in
    the population. If the population is constantly 
    harvested it is modeled by the differential equation
    \[
        P' = .02P - k
    \]
    Discuss the role of the parameter $k > 0$. Prove that 
    if $k$ is very large the population will go extinct 
    while for $k$ small the population will thrive. Find
    the value of $k$ that separates these regimes in terms
    of the initial population and the growth rate.
\end{problem}

\begin{solution}
    \vfill
\end{solution}
\clearpage

\begin{problem}[C6]
    Let $f(x, y)$ be defined for all $(x, y)\in\RR^2$,
    continuous and continuously differentiable in the
    variable $y$. Assuming that $f(x, y)y \leq 0$ for
    all $(x, y)$, find the solution to $y' = f(x, y)$, 
    $y(-.333) = 0$ (Hint: Compute $f(x, 0)$). Sketch
    the graphs of the solutions to this differential 
    equation that satisfy $y(0) = 1$, and $y(0) = -1$.
\end{problem}

\begin{solution}
    \vfill
\end{solution}
\clearpage

\begin{problem}[C7]
    Justify why the differential equation $y' = 
    \sin(x^2+y^3)$ is not separable, linear, or exact.
    Use a computer spreadsheet to approximate the solution
    given by Euler’s method to this differential that
    satisfies $y(0) = 1$ in the interval $[0, 1]$ with 
    mesh of size $h = .01$, and with mesh size $h = .001$.
    Find $y(.k)$ for $k = 1,\dots,10$ for each approximation. 
    Use the graphing feature of the spreadsheet to obtain 
    in the same caption the graphs of both solutions.
\end{problem}

\begin{solution}
    \vfill
\end{solution}
\clearpage

\begin{problem}[C8]
    The solution to $y'=(1+x^4)y^2$, $y(0) = 1$ blows up in
    $(0,\infty)$. Use a computer spreadsheet to estimate
    the blow up time using three Euler approximations to the 
    solution. Find $y(.k)$ for $k = 1,\dots,10$ for each
    approximation. Separate the variables in this equation and 
    find the solution. Compare the estimate for the blow up 
    time obtained from the numerical approximations and the 
    one given by the actual solution.
\end{problem}

\begin{solution}
    \vfill
\end{solution}
\clearpage

\end{document}
